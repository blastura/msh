\documentclass[a4paper, 12pt]{article}
\usepackage[swedish]{babel}
\usepackage[utf8]{inputenc}
\usepackage{verbatim}
\usepackage{fancyhdr}
\usepackage{graphicx}
\usepackage{parskip}
\usepackage{minitoc}

% Include pdf with multiple pages ex \includepdf[pages=-, nup=2x2]{filename.pdf}
\usepackage[final]{pdfpages}
% Place figures where they should be
\usepackage{float}

% Float for text
\floatstyle{ruled}
\newfloat{xml}{H}{lop}
\floatname{xml}{XML}

% vars
\def\title{msh}
\def\preTitle{Laboration 2}
\def\kurs{Systemprogrammering, VT-09}

\def\namn{Anton Johansson}
\def\mail{dit06ajn@cs.umu.se}
\def\pathtocode{$\sim$dit06ajn/edu/sysprog/lab1}

\def\handledareEtt{Johan Eliasson, johane@cs.umu.se}
\def\handledareTva{Johan Granberg, johang@cs.umu.se}

\def\inst{datavetenskap}
\def\dokumentTyp{Laborationsrapport}

\begin{document}
\begin{titlepage}
  \thispagestyle{empty}
  \begin{small}
    \begin{tabular}{@{}p{\textwidth}@{}}
      UMEÅ UNIVERSITET \hfill \today \\
      Institutionen för \inst \\
      \dokumentTyp \\
    \end{tabular}
  \end{small}
  \vspace{10mm}
  \begin{center}
    \LARGE{\preTitle} \\
    \huge{\textbf{\kurs}} \\
    \vspace{10mm}
    \LARGE{\title} \\
    \vspace{15mm}
    \begin{large}
        \namn, \mail \\
        \texttt{\pathtocode}
    \end{large}
    \vfill
    \large{\textbf{Handledare}}\\
    \mbox{\large{\handledareEtt}}
    \mbox{\large{\handledareTva}}
  \end{center}
\end{titlepage}

\newpage
\mbox{}
\vspace{70mm}
\begin{center}
% Dedication goes here
\end{center}
\thispagestyle{empty}
\newpage

\pagestyle{fancy}
\rhead{\today}
\lhead{\namn, \mail}
\chead{}
\lfoot{}
\cfoot{}
\rfoot{}

\cleardoublepage
\newpage
\dosecttoc 
\tableofcontents
\cleardoublepage

\rfoot{\thepage}
\pagenumbering{arabic}

\section{Problemspecifikation}\label{sec:problemspecifikation}
Denna laboration gick ut på att skriva en enkel variant ett UNIX-skal
som till exempel
\textit{bash}\footnote{http://tiswww.case.edu/php/chet/bash/bashtop.html}.
Det vill säga ett program som hjälper till vid exekvering av andra
program.

Problemspecifikation finns i original på sidan:\\
\begin{footnotesize}
\verb!http://www.cs.umu.se/kurser/5DV004/VT09/labbar/lab2/index.html!
\end{footnotesize}

\section{Användarhandledning}\label{sec:anvandarhandledning}
Källkoden till programmet finns i katalogen
\verb!~dit06ajn/edu/5DV004/lab2! och kompileras genom att i en
kommandopromt från denna katalog köra följande kommandon:

\verb!make!

Detta i sin tur kommer att kompilera filerna \verb!mdu.c! och
\verb!pathalloc.c! från katalogen \verb!src! och skapa en körbar fil,
\verb!mdu! i underkatalogen \verb!bin!. Kommandot som anropas från
\verb!make! är:

\verb!gcc -Wall -g -o ./bin/mdu ./src/mdu.c ./src/pathalloc.c!

Programmet körs med enligt syntax:

\verb!mdu [-s|-a] [dir ...]!

Flaggan \verb!-s! gör så att enbart storleken på angivna kataloger
skrivs ut, inte alla underkataloger. Flaggan \verb!-a!  skriver ut
storleken på alla filer, inte enbart kataloger. Exempelkörning:

\begin{verbatim}
$ bin/mdu -s .. ~dit06ajn/edu
292     ..
276120  /home/dit06/dit06ajn/edu
$
\end{verbatim}

Ovan kommando skriver ut den allokerade storleken av de två angivna
katalogerna \verb!..! och \verb!~dit06ajn/edu! i kilobyte.

% \bibliographystyle{alpha}
% \bibliography{books.bib}

\section{Algoritmbeskrivning}\label{sec:algoritmbeskrivning}

\section{Testkörningar}\label{sec:testkorningar}

\section{Lösningen begränsningar}\label{sec:losningensbegransningar}

\section{Problem och reflektioner}\label{sec:problemochreflektioner}

\newpage
\appendix
\pagenumbering{arabic}
\section{Bilagor}\label{Bilagor}
% Källkoden ska finnas tillgänglig i er hemkatalog
% ~/edu/apjava/lab1/. Bifoga även utskriven källkod.
Härefter följer utskrifter från källkoden och andra filer som hör till
denna laboration.

\newpage
\subsection{mdu.c}\label{mdu.c}
\begin{scriptsize}
  \verbatiminput{../src/msh.c}
\end{scriptsize}
\end{document}
