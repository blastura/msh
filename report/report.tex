% TODO: Vad händer om signaler körs väldigt länge och någon råkar ta
% över ett pid, då kommer pidArray fortfarande ha kvar ett pidNummer
% och man kan komma att avsluta fel process.

% TODO: Måste skriva om varför interna kommandon måste vara interna,
% det är för att pwd och variabler inte kan ändras utanför den egna
% processen.

\documentclass[a4paper, 12pt]{article}
\usepackage[swedish]{babel}
\usepackage[utf8]{inputenc}
\usepackage{verbatim}
\usepackage{fancyhdr}
\usepackage{graphicx}
\usepackage{parskip}
\usepackage{minitoc}

% Include pdf with multiple pages ex \includepdf[pages=-, nup=2x2]{filename.pdf}
\usepackage[final]{pdfpages}
% Place figures where they should be
\usepackage{float}

% Float for text
\floatstyle{ruled}
\newfloat{xml}{H}{lop}
\floatname{xml}{XML}

% vars
\def\title{msh}
\def\preTitle{Laboration 2}
\def\kurs{Systemprogrammering, VT-09}

\def\namn{Anton Johansson}
\def\mail{dit06ajn@cs.umu.se}
\def\pathtocode{$\sim$dit06ajn/edu/5DV004/lab2}

\def\handledareEtt{Johan Eliasson, johane@cs.umu.se}
\def\handledareTva{Johan Granberg, johang@cs.umu.se}

\def\inst{datavetenskap}
\def\dokumentTyp{Laborationsrapport}

\begin{document}
\begin{titlepage}
  \thispagestyle{empty}
  \begin{small}
    \begin{tabular}{@{}p{\textwidth}@{}}
      UMEÅ UNIVERSITET \hfill \today \\
      Institutionen för \inst \\
      \dokumentTyp \\
    \end{tabular}
  \end{small}
  \vspace{10mm}
  \begin{center}
    \LARGE{\preTitle} \\
    \huge{\textbf{\kurs}} \\
    \vspace{10mm}
    \LARGE{\title} \\
    \vspace{15mm}
    \begin{large}
        \namn, \mail \\
        \texttt{\pathtocode}
    \end{large}
    \vfill
    \large{\textbf{Handledare}}\\
    \mbox{\large{\handledareEtt}}
    \mbox{\large{\handledareTva}}
  \end{center}
\end{titlepage}

\newpage
\mbox{}
\vspace{70mm}
\begin{center}
% Dedication goes here
\end{center}
\thispagestyle{empty}
\newpage

\pagestyle{fancy}
\rhead{\today}
\lhead{\namn, \mail}
\chead{}
\lfoot{}
\cfoot{}
\rfoot{}

\cleardoublepage
\newpage
\dosecttoc 
\tableofcontents
\cleardoublepage

\rfoot{\thepage}
\pagenumbering{arabic}

\section{Problemspecifikation}\label{sec:problemspecifikation}
Denna laboration gick ut på att skriva en enkel variant ett UNIX-skal
som till exempel
\textit{bash}\footnote{http://tiswww.case.edu/php/chet/bash/bashtop.html}.
Det vill säga ett program som hjälper till vid exekvering av andra
program.

Problemspecifikation finns i original på sidan:\\
\begin{footnotesize}
\verb!http://www.cs.umu.se/kurser/5DV004/VT09/labbar/lab2/index.html!
\end{footnotesize}

\section{Användarhandledning}\label{sec:anvandarhandledning}
Källkoden till programmet finns i katalogen
\verb!~dit06ajn/edu/5DV004/lab2! och kompileras genom att i en
kommandopromt från denna katalog köra följande kommandon:

\begin{verbatim}
salt:dit06ajn:~/edu/5DV004/lab2$ make
salt:dit06ajn:~/edu/5DV004/lab2$ make install
\end{verbatim}

Detta i sin tur kommer att kompilera de nödvändiga filerna i katalogen
\verb!src! och placera en körbar fil \verb!msh! i katalogen \verb!bin!. och
\verb!pathalloc.c! från katalogen \verb!src! och skapa en körbar fil,
\verb!msh! i underkatalogen \verb!bin!. Programmet körs från katalogen
\verb!bin! enligt syntax:

\verb!./msh [scriptfiles...]!

Ovan kommando utan parameter för skriptfil kommer att skriva ut en
prompt \textit{msh\%} för användaren. Efter denna prompt kan
användaren skriva nya kommando som i de flesta andra skalprogram.
Exempel:
\begin{verbatim}
msh% ls ..
bin  Makefile  report  src
\end{verbatim}

% \bibliographystyle{alpha}
% \bibliography{books.bib}

\section{Systembeskrivning}\label{sec:systembeskrivning}
Skalprogrammet startas upp genom \textit{main}-funktionen i filen
\textit{msh.c}. Denna metod läser av om det finns några parametrar som
ska köras i skalet. Finns sådana parametrar läses dessa in och skickas
som en ström till metoden \textit{shell}, finns inget skript i
parametrarna skickas strömmen \textit{stdin} med. Metoden
\textit{shell} ta även emot en andra parameter som avgör om skalet ska
köras i skriptläge eller i ett interaktivt läge. Skillnaden
mellan dessa lägen är att i skriptläget skrivs inga promptar ut.

Metoden \textit{shell} har en loop som läser av indata i form av hela
rader från den medskickade strömmen. När en hel rad har skickats in
kommer denna rad att gås igenom av metoden \textit{expand} som byter
ut varje förekomst av ''\$(XXX)'' med mostsvarande miljövariabel för
''XXX''. Exempelvis kommer raden 'echo \$(HOST)' kommer att bli 'echo
salt' om man är inloggad på maskinen salt.cs.umu.se.

När alla miljövariabler bytts ut kommer resultatet att avkodas med
hjälp av den givna metoden \textit{parse}. Metoden \textit{parse} ser
till att raden avkodas så att rader som \verb!ls | sort > srt.txt!
kodas som två kommandon \textit{ls} och \textit{sort} och att utdata
från kommandot \textit{sort} ska hamna i en nyskapad fil som heter
\textit{srt.txt}. Strukturen för ett kommando \textit{command} finns
att se i filen \textit{parser.h}.

När information om de olika kommandona finns avkodade är skalet redo
att exekvera de givna metoderna, detta görs av metoden
\textit{doCommands(command comLine[], int nrOfCommands)}. Se mer
information om denna metod i algoritmbeskrivningen, avsnitt
\ref{sec:doCommand}. Efter att kommandona är utförda kommer
skalprogrammet att vänta på att nästa rad finns tillgängligt för
läsning, det vill säga vid radbrytningar i strömmen som avläses.

\subsection{Interna kommandon}\label{sec:internakommandon}
Eftersom en process enbart kan ändra sina egna miljövariabler och inte
andra processers så måste vissa kommandon som bör finnas i ett skal
implementeras på ett annat sätt än andra vanliga program som
exekveras. Dessa kommandon i denna laboration är \verb!cd!, som ändrar
skalets arbetskatalog och kommandot \verb!set var=value!, som sätter
miljövariabeln \textit{var} till värdet \textit{value}. Dessa
ändringar görs med funktionerna \textit{chdir()} och
\textit{setenv()}, skulle dessa metoder köras i en barnprocess skulle
ändringarna inte gälla för föräldraprocessen som kör skalet.

\section{Algoritmbeskrivning}\label{sec:algoritmbeskrivning}
Följande avsnitt ger en beskrivning av hur olika dela i systemet
fungerar.

\subsection{doCommand}\label{sec:doCommand}
Metoden \textit{doCommand} tar hand om exekveringen av kommandon som
skalet läser in. En kombination av kommandon kan till exempel vara
exekvering av \verb!ls | sort | wc!, här ska \textit{ls} köras och dess
utdata ska kopplas till indatan för \textit{sort} som i sin tur
kopplar sin utdata till indata för \textit{wc}. Hantering
vidaresändning av standard indata och standard utdata görs med hjälp
av så kallade pipor som skapas med UNIX-metoden \textit{pipe}.

För varje kommando som ska exekveras skapas en ny pipa och en ny
process skapas genom systemanropet \textit{fork}. Om kommandot är det
första i en kedja så kopplas utdata från kommandot till pipan.
Nästkommande kommando kopplar sin indata till samma pipa och om fler
kommandon följer kopplas utdata till en ny pipa. Det sista kommandot i
en sådan här kedja kopplar inte om sin utdata vilket resulterar att
sista kommandot kommer att skriva till standard utdata-ström.

Koppling av standard utdata och indata gör med hjälp av funktionen
\textit{dupPipe}, implementerad i filen \textit{execute.c}, som i sin
tur anropar systemanropet \textit{dup2}. För att omdirigera standard
utdata och indata till och från filer används kommandot
\textit{redirect}, implementerat samma fil som \textit{dupPipe}.
Metoden \textit{redirect} öppnar aktuell fil och kopierar den till
standard utdata-strömmen eller indata-strömmen med hjälp av
systemanropet \textit{dup2}.

Efter att barnen skapats ställer sig föräldraprocessen och väntar på
de processID:n som barnen har fått. När alla barn har avslutats eller
en signal \textit{SIGINT} fångats avslutas föräldraprocessen och
\textit{doCommand} avslutas.

\section{Testkörningar}\label{sec:testkorningar}
Följande avsnitt demonstrerar ett antal kommandon som är möjliga att
köra från skalprogrammet, samt resulterande utdata. All exekvering med
prompten \textit{msh\%} sker från skalprogrammet.

\subsection{Enkla kommandon}\label{sec:enkla-kommandon}
Följande kommandon demonstrerar simpla kommandon och dess utdata:

\begin{scriptsize}
\begin{verbatim}
msh% ls -la
total 0
drwxr-xr-x   2 anton  staff   68 12 Maj 01:32 .
drwxr-xr-x  10 anton  staff  340 12 Maj 01:29 ..
\end{verbatim}
\end{scriptsize}

\begin{scriptsize}
\begin{verbatim}
msh% mkdir mapp1
msh% ls
mapp1
\end{verbatim}
\end{scriptsize}

\subsection{Pipor}\label{sec:pipor}
Följande kommandon demonstrerar användandet av pipor mellan kommandon:
\begin{scriptsize}
\begin{verbatim}
msh% ls
mapp1
msh% touch a            
msh% touch b
msh% ls
a     b     mapp1
msh% ls | sort
a
b
mapp1
msh% ls | sort | wc -l
       3
\end{verbatim}
\end{scriptsize}

\subsection{Omdirigering}\label{sec:omdirigering}
För att omdirigera utdata till en fil:
\begin{scriptsize}
\begin{verbatim}
msh% ls > lsFil.txt
msh% cat lsFil.txt
a
b
lsFil.txt
mapp1
\end{verbatim}
\end{scriptsize}

För att omdirigera data från en fil till indata för ett kommando:
\begin{scriptsize}
\begin{verbatim}
msh% sort -r < lsFil.txt
mapp1
lsFil.txt
b
a
\end{verbatim}
\end{scriptsize}

Som förra exemplet fast utdata dirigeras in i en fil:
\begin{scriptsize}
\begin{verbatim}
msh% sort -r < lsFil.txt > revSortedLsFil.txt
msh% cat revSortedLsFil.txt
mapp1
lsFil.txt
b
a
\end{verbatim}
\end{scriptsize}

\subsection{Signalhantering}\label{sec:signalhantering}
Följande exempel visar ett kommando som hänger sig och som avbryts
genom att signalen \textit{SIGINT} skickas genom att användaren
trycker Ctrl-C.
\begin{scriptsize}
\begin{verbatim}
msh% sort
^Cmsh% 
\end{verbatim}
\end{scriptsize}

\section{Lösningen begränsningar}\label{sec:losningensbegransningar}
Implementationen så som det är nu är inte väldigt användbar för annat
än ett enkelt exempel på hur ett skal i all simpelhet skulle kunna
implementeras. Det finns en rad småbuggar som skulle behöva ordnas,
till exempel avslutas programmet om användaren trycker ENTER utan att
först skrivit in någonting. Vid testkörningar på en Solaris-maskin
upptäcktes även att vissa funktioner och konstanter inte fanns i samma
inkluderade .h filer som för på Mac OSX, de filer som skiljer är
nu inkluderade på alla system vilket kanske inte är helt optimalt.

Noterat är även att implementationen inte fungerar helt felfritt på en
Sun Solaris maskin. Det verkar som att parametrar till kommandon inte
skickas med på rätt sätt.

\section{Problem och reflektioner}\label{sec:problemochreflektioner}
Det var väldigt svårt att få överblick över hur pipor tillsamman med
systemanropet \textit{fork} fungerade och vad som krävdes för att en
barnprocess verkligen skulle avslutas. Det tog väldigt lång tid att
felsöka fram att en fildeskriptor som glömt stängas vilket gjorde att
skalet inte kunde hantera fler än en pipa. På grund av oerfarenhet i
felsökning av C-program har även tiden inte räckt till för att
implementera alla krav enligt lobarationsspecifikationen. Dock känns
det som att resterande funktionalitet inte är den svåra biten av
laborationen, men man vet ju aldrig innan man försökt...

\newpage
\appendix
\pagenumbering{arabic}
\section{Bilagor}\label{Bilagor}
% Källkoden ska finnas tillgänglig i er hemkatalog
% ~/edu/apjava/lab1/. Bifoga även utskriven källkod.
Härefter följer utskrifter från källkoden och andra filer som hör till
denna laboration.

\newpage
\subsection{Makefile}\label{Makefile}
\begin{scriptsize}
  \verbatiminput{../src/Makefile}
\end{scriptsize}

\subsection{execute.c}\label{execute.c}
\begin{scriptsize}
  \verbatiminput{../src/execute.c}
\end{scriptsize}

\subsection{execute.h}\label{execute.h}
\begin{scriptsize}
  \verbatiminput{../src/execute.h}
\end{scriptsize}

\subsection{msh.c}\label{msh.c}
\begin{scriptsize}
  \verbatiminput{../src/msh.c}
\end{scriptsize}

\subsection{parser.h}\label{parser.h}
\begin{scriptsize}
  \verbatiminput{../src/parser.h}
\end{scriptsize}

\subsection{sighant.c}\label{sighant.c}
\begin{scriptsize}
  \verbatiminput{../src/sighant.c}
\end{scriptsize}

\subsection{sighant.h}\label{sighant.h}
\begin{scriptsize}
  \verbatiminput{../src/sighant.h}
\end{scriptsize}

\end{document}
